\documentclass{article}

\usepackage{stmaryrd}
\usepackage{graphicx}
\usepackage{listings}
\usepackage{amsmath}
\usepackage{amssymb}
\usepackage{amstext} 
\usepackage{bbm}
\usepackage{array}   
\usepackage{mathrsfs}
\usepackage{longtable}
\usepackage[margin=0.8in]{geometry}
\usepackage{parskip}
\usepackage[utf8]{inputenc}

\newcommand{\HMeas}[1]{\mathbb{M}(#1)}
\newcommand{\HLOsem}[1]{\llbracket #1 \rrbracket}
\newcommand{\Hhl}[1]{\big[#1\big]}
\newcommand{\Hdo}[1]{\text{\textbf{do}\{} #1 \text{\}}}
\newcommand{\arr}[0]{\rightarrow}
\newcommand{\Ht}[4]{<\hspace{-2pt}\scalebox{1.16}{$\triangleleft$}\, #1\, #2\, #3\, #4}
\newcommand{\real}[0]{\mathbb{R}}
\newcommand{\integer}[0]{\mathbb{Z}}
\renewcommand{\inf}[0]{\infty}
\renewcommand{\d}[0]{\raisebox{1pt}{\,\,$\scriptstyle\wedge$\,}}
\newcommand{\circled}[1]{\text{\textcircled{\scalebox{0.86}{$#1$}}}}
\newcommand{\cmp}[0]{{\,\, \scriptstyle \circ \,\,}}
\newcommand{\der}[0]{\text{d}}
\newcommand{\Hsubs}[2]{\left. #1 \right\rvert_{#2}}

\newcolumntype{L}{>{$}l<{$}}
\newcolumntype{R}{>{$}r<{$}}
\newcolumntype{C}{>{$}c<{$}}

\begin{document}

\renewcommand{\arraystretch}{1.23}
%% \[
%% \begin{array}{rcllr}
\begin{longtable}{RCLp{0.35\linewidth}R}
\HLOsem{\_} & : & \HMeas{A} \arr ((A \arr \real) \arr \real) \\
\HLOsem{\_} & = & \text{LO interpretation} \\
m   & : & \HMeas{A,T} \\ 
m   & = & \Hdo{ h ; a \gets w; M } & input to disint \\ 
\hat{m} & = & \Hdo{a \gets \mu; t \gets k\, a; \text{return}(a,t) } & $m$ to $\hat{m}$ by algebraic manipulation \\
\mu & : & \HMeas{A} \\ 
    & \in & \{\text{lebesgue,counting,dirac}\} & in concrete code, $\mu$ is determined by $A$ (type directed) \\
k   & : & A \arr \HMeas{T} \\
    & = & \text{the interesting thing} \\
\HLOsem{\hat{m}} & = & \lambda f . \HLOsem{\mu} \left( \lambda a . \HLOsem{k\, a} (\lambda t . f(a,t)) \right) & defn of $\HLOsem{\_}$. Given $\hat{m}$, solve for $k$\\
\hat{m} & = & \Hdo{a \gets \mu;\Ht{w}{a}{\bar{M}}{h} } \\
\text{where} && 
 \begin{array}{rcl}
 \bar{M} &=& \lambda h' . \Hdo{h' ; M} \\ 
 \Ht{\_}{\_}{\_}{\_} &=& \text{constrain outcome ???} \\
 \end{array} \\
%%%%%%%%%%%%%%%%%%%
\multicolumn{3}{L}{\text{Target: }} \\ 
\HLOsem{\hat{m}} & =&  \lambda f . \int_{a = (- \inf, \inf)}  \HLOsem{k\, a} (\lambda t . f(a,t)) \d \text{d}a \\ 
%%%%%%%%%%%%%%%%%%%
\multicolumn{3}{L}{\text{Assuming } \mu = \text{lebesgue}} \\ 
\HLOsem{\hat{m}} & =&  \lambda f . \int_{a = (- \inf, \inf)} \HLOsem{\Ht{w}{a}{\bar{M}}{h}}(f) \d \text{d}a \\ 
\text{where} & & \multicolumn{2}{L}{\begin{array}{rcl} \HLOsem{\bar{M}h'} &=& \HLOsem{h'} \Hhl{ \HLOsem{M}(f) } \end{array}} & (a) \\
%%%%%%%%%%%%%%%%%%%
\multicolumn{3}{L}{\text{General form}} \\ 
\HLOsem{\hat{m}} & =&  \lambda f . \HLOsem{h} \Hhl{ \HLOsem{w}(\lambda a . \HLOsem{M}(f)) } \\ 
\text{where} && \multicolumn{2}{L}{\begin{tabular}{RCp{0.7\linewidth}}
\HLOsem{h} & \text{is} & an expression with holes, such that substituting a PL (patently linear) -in-$f$ expression gives a PL-in-$f$ expression.
\end{tabular}}\\
\text{assuming} && \multicolumn{2}{L}{\begin{array}{rcl}
a & \in & \text{free-vars}( \HLOsem{M} ) \\
\text{free-vars}(h) & \subseteq & \text{free-vars} ( \HLOsem{w} ) \cup \text{free-vars} ( \HLOsem{M} ) 
\end{array}}\\\\
%%%%%%%%%%%%%%%%%%%
\multicolumn{3}{l}{Then case split on $w$:} \\ 
\circled{1} \quad w & = & \text{lebesgue} \\ 
\HLOsem{\hat{m}}(f)  & =&  \HLOsem{h} \Hhl{ \int_{a = (-\inf, \inf) } \HLOsem{M}(f)  \d \text{d}a } \\ 
                     & =&  \int_{a = (-\inf, \inf) } \HLOsem{h} \Hhl{ \HLOsem{M}(f) } \d \text{d}a
 & because of linearity/Tonelli/$h$ is nice enough \\ 
\multicolumn{4}{L}{\text{Thus define } \Ht{\text{lebesgue}}{a}{\bar{M}}{h'} = \bar{M} h' \text{ where we used eqn$_{(a)}$}  } \\ 
%%%%%%%%%%%%%%%%%%%
\\\circled{2} \quad w & = & \text{Normal} \, \mu \, \sigma & PDF is $\hat{N_x}$ \\ 
\HLOsem{\hat{m}}(f)  & =&  \HLOsem{h} \Hhl{ \int_{a = (-\inf, \inf) } \hat{N}_a  \cdot \HLOsem{M}(f) \d \text{d}a } \\ 
                     & =&  \int_{a = (-\inf, \inf) } \HLOsem{h} \Hhl{ \hat{N}_a \cdot \HLOsem{M}(f) } \d \text{d}a 
 & \text{linearity/Tonelli/$h$ nice} \\ 
    & =&  \int_{a = (-\inf, \inf) } \bar{M}\Big( \HLOsem{h} \Hhl{ \hat{N}_a \cdot \Hhl{\,} } \Big)  \d \text{d}a 
  & \text{???} \\ 
\multicolumn{4}{L}{\text{Thus define } \Ht{(\text{Normal} \, \mu \, \sigma)}{a}{\bar{M}}{h'} = \bar{M} (h' ; \text{factor}\, \hat{N}_{\mu,\sigma})  } \\
%%%%%%%%%%%%%%%%%%%
\\\circled{3} \quad w & = & \text{counting} \\
\HLOsem{\hat{m}}(f)  & =&  \HLOsem{h} \Hhl{ \sum_{a = (-\inf, \inf) } \HLOsem{M}(f) } \\ 
  & =&  \sum_{a = (-\inf, \inf) } \HLOsem{h} \Hhl{  \HLOsem{M}(f) } 
 & linearity/switch/$h$ nice \\ 
  & =&  \text{FAIL} \\
\multicolumn{4}{L}{\text{But we ``knew'' that the type of $w$ is $\HMeas{\integer}$ and we are in $\HMeas{\real}$}}\\
%%%%%%%%%%%%%%%%%%%
\\\circled{4} \quad w & = & \text{Dirac} \, d & $h \vdash d : \real$ \\
\HLOsem{\hat{m}}(f)  & =&  \HLOsem{h} \Hhl{ \Hsubs{ \HLOsem{M}(f) }{a=d} } \\
\multicolumn{4}{l}{
\begin{tabular}{ll}
case 1: & $d$ is a constant, likely impossible? \\ 
case 2: & $d$ is a var bound in $h$, see POPL backup slides. case \emph{return x}. see also outline $+$ proofs (???) \\ 
case 3: & $d = g\, x$ where $x$ bound in $h$ \\
 & $ \Ht{\HLOsem{\text{Dirac } (g\, x)}}{a}{c}{(h_1 ; x \gets w_1; h_2)} = \Ht{w_1}{(g^{-1}\, a)}{?}{?} $ \\ 
 & worry: $g^{-1}$ may involve (????) \\
\end{tabular}}\\
%%%%%%%%%%%%%%%%%%%
\\\circled{5} \quad w & = & \text{return} \, | e | \\
\HLOsem{\hat{m}}(f)  & =&  \HLOsem{h} \Hhl{ \Hsubs{ \HLOsem{M}(f) }{a=|e|} } \\
 & =&  \HLOsem{h} \Hhl{ \text{if } e \geq 0 \text{ then } \Hsubs{ \HLOsem{M}(f) }{a=e} \text{ else } \Hsubs{ \HLOsem{M}(f) }{a=-e} }
  & \text{?} \\
 & =&  \HLOsem{h} \Hhl{ \Hsubs{ \HLOsem{ \Hdo{a \gets \text{return } | b |; M }(f) } }{b=e} } \\
 & =&  \int_{b=(-\inf,\inf)} \HLOsem{ \Ht{(\text{return } e)}{b}{ \Hsubs{M}{a=|b|} }{ h }  }(f) \d \text{d}b 
  & \text{by induction} \\
 & =&  \int_{b=(-\inf,\inf)} \Hsubs{ \HLOsem{ \Ht{(\text{return } e)}{b}{ \bar{M} }{ h }  }(f) }{a=|b|} \d \text{d}b 
  & \text{substitution commutative} \\ 
 & =&  \int_{b=(-\inf,0)} \Hsubs{ \HLOsem{ \Ht{(\text{return } e)}{b}{ \bar{M} }{ h }  }(f) }{a=-b} \d \text{d}b \\
 & +&  \int_{b=(0,\phantom{-}\inf)}  \Hsubs{ \HLOsem{ \Ht{(\text{return } e)}{b}{ \bar{M} }{ h }  }(f) }{a=\phantom{-}b} \d \text{d}b \\
 & =&  \int_{a=(0,\inf)} \Hsubs{ \HLOsem{ \Ht{(\text{return } e)}{b}{ \bar{M} }{ h }  }(f) }{b=-a} \d \text{d}a
  & \text{change of variables from $b$ to $-a$} \\
 & +&  \int_{b=(0,\inf)}  \Hsubs{ \HLOsem{ \Ht{(\text{return } e)}{b}{ \bar{M} }{ h }  }(f) }{a=\phantom{-}b} \d \text{d}b \\
 & =&  \int_{a=(0,\inf)} { \HLOsem{ \Ht{(\text{return } e)}{-\hspace{-3pt}a}{ \bar{M} }{ h }  }(f) }{} \d \text{d}a \\
 & +&  \int_{a=(0,\inf)}  { \HLOsem{ \Ht{(\text{return } e)}{\phantom{-}\hspace{2pt}a}{ \bar{M} }{ h }  }(f) }{} \d \text{d}a
  & \text{$\hat{M}$ is $b$-free} \\ %% Latex sucks at aligning `-`, there should be a special prefix char 
 & =&  \int_{a=(0,\inf)} \HLOsem{ 
    \cdots %% \Ht{(\text{return } e)}{-\hspace{-3pt}a}{ \bar{M} }{ h } 
    \,\,\circled{+}\,
    \cdots  %% \Ht{(\text{return } e)}{a}{ \bar{M} }{ h }
    }(f) \d \text{d}a \\
 & =&  \int_{a=(-\inf,\inf)} \mathbbm{1}_{[0,\inf)}(a)  \cdot \HLOsem{ \cdots 
    %% \Ht{(\text{return } e)}{-\hspace{-3pt}a}{ \bar{M} }{ h } \,\,\circled{+}\,
    %% \Ht{(\text{return } e)}{a}{ \bar{M} }{ h }
    }(f) \d \text{d}a \\
 & =&  \int_{a=(-\inf,\inf)} \HLOsem{ \text{guard } [0, \inf)\, a; \cdots 
    %% \Ht{(\text{return } e)}{-\hspace{-3pt}a}{ \bar{M} }{ h } \,\,\circled{+}\,
    %% \Ht{(\text{return } e)}{a}{ \bar{M} }{ h }
    }(f) \d \text{d}a \\
%%%%%%%%%%%%%%%%%%%
\\\multicolumn{4}{l}{And now onto $\frac{\der}{\der a}$! Definition 2. Let:} \\ 
\hat{m} & = & \Hdo{x \gets \mu; y \gets k\, x; \text{return}(x,y) } \quad \text{where } \mu = \text{lebesgue} \\
\multicolumn{4}{l}{Show that:} \\
\HLOsem{k\, a}(f)  & =& \displaystyle\frac{\der}{\der a} \left( \HLOsem{\hat{m}} \left( \lambda (x,y) . 
   \left\{ \begin{array}{lr} f(y) & x \leq a \\ 0 & x > a \end{array} \right.
     \right)  \right) \\
  & =& \displaystyle\frac{\der}{\der a} \left( \int_{x=(-\inf,\inf)} \HLOsem{{k\, x}} \left( \lambda y . 
   \left\{ \begin{array}{lr} f(y) & x \leq a \\ 0 & x > a \end{array} \right.
     \right) \d \der x \right) & defn. of $\HLOsem{\cdot}$ \\
  & =& \displaystyle\frac{\der}{\der a} \left( \int_{x=(-\inf,\inf)} 
   \left\{ \begin{array}{lr} \HLOsem{k\, x}(f) & x \leq a \\ \HLOsem{k\, x}(\lambda \_ . 0) & x > a \end{array} \right.
       \d \der x \right) & piecewise rules \\
  & =& \displaystyle\frac{\der}{\der a} \left( \int_{x=(-\inf,a)} 
     \HLOsem{k\, x}(f)
       \d \der x \right) & linearity $T(\lambda \_ . 0) = 0$ \\
  & =&  \HLOsem{k\, a}(f) & Leibniz integral rule ($k$ depends on $a$?)\\
\end{longtable}

\end{document}
