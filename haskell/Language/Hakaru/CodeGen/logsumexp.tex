\documentclass[12pt]{article}
\usepackage[margin=1.2in]{geometry}
\usepackage{amsmath}
\usepackage{amssymb}
\usepackage{tikz}
\usepackage{tikz-qtree}
\usepackage{verbatim}
\usepackage{algorithm,caption}

% \usepackage{natbib}
% \twocolumn

\newcommand{\ind}{\hspace*{1em}}
\newcommand{\kword}[1]{{\bf #1}}
\newcommand{\algblk}[2]{
  \begin{center}
    \begin{minipage}[c]{22em}
      \begin{algorithm}[H]
        \caption*{{\bf Algorithm} #1}
        #2
      \end{algorithm}
    \end{minipage}
  \end{center}
}


\title{Adding Probabilities in the Hakaru to C Compiler (HKC)}
\author{Zach Sullivan}
\date{July 30, 2016}

\begin{document}
\maketitle

\section{Probabilities in Hakaru}

Because Hakaru is a probabilistic language, we provide a type just for
probabilities. The type {\tt prob} has extra safety from underflow. The main way
we accomplish this is by storing its value as double precision floating point
numbers in the log-domain. We can do basic arithmetic on our probabilities.

\section{The ``LogSumExp Trick''}
Because our probability types are stored in the log-domain, we need to compute
\begin{displaymath}
{\rm LSE}(x) = \log\Bigg(\sum_{i=0}^{n}{e^{x_i}}\Bigg)
\end{displaymath}
often called ``LogSumExp.'' The advantage of storing probabilities in the
log-domain is lost if we simply remove them from the log-domain before doing
calculations on them. We would also add a bunch of {\tt log} and {\tt exp}
operations.

\begin{align*}
{\rm LSE^\prime}(x) &= \hat x + \log\Bigg(\sum_{i=0}^{n}{e^{x_i-\hat x}}\Bigg)\\
{\rm\bf where}&\ \hat x = {\rm max}(x)
\end{align*}

This trick preserves the value of the largest of the numbers being summed.


\section{Summation over Probabilities}

For summation over probabilities a different technique for LogSumExp will need
to be use. Since arity will not be known at runtime we need to have a function
that operates over an array.

This algorithm follows directly from the math:

\algblk{LogSumExp for Summation}
{
  \kword{input:} an index production function
  $f : \mathbb{N} \rightarrow \mathbb{R}^+$ and a size $n \in \mathbb{N}$

  \kword{output:} summation of array produced by $f$ in log-space\\

  $A[0] \leftarrow f(0)$\\
  $m \leftarrow A[0]$\\
  $s \leftarrow 0$\\
  \kword{for} $i = 0,1,...,n-1$\kword{:}\\
  \ind $A[i] \leftarrow f(i)$\\
  \ind \kword{if} $m < A[i]$ \kword{:}\\
  \ind \ind $m \leftarrow A[i]$\\
  \kword{for} $i = 0,1,...,n-1$\kword{:}\\
  \ind \ind $s \leftarrow s + \exp(A[i]-m)$\\
  \kword{return} $m + \log(s)$
}

It scales in space at $O(n)$ and in time at $O(2n)$. We generate a function
for this operation and call it when we summate over probabilities. It differs
from summation over {\tt nat}, {\tt int}, and {\tt real} which are done in
constant time.

\section{$n$-ary Probability Summation}

Algorithmically, $n$-ary operations are the same operation as summation over an
array. However, it requires different word for code generation. For summation
we do not know that arity at runtime. For $n$-ary operations, we do.


\subsection*{Max Comparison Tree}
LogSumExp safety creates a particular challenge when generating code for our
compiler. HKC compiles Hakaru to C, where we have no max function that works
on an arbitrary number of arguments. The solution is to create a tree of
comparisons using C's ternary conditional expression to find the maximum of $n$
number of arguments.

Here is an example of a max comparison tree when the length of array $x$ is
4.

\begin{center}
\Tree [ .{$x_0 > x_1$}
        [ .{$x_0 > x_2$}
          [ .{$x_0 > x_3$}
            % {$x_0$}
            % {$x_3$}
          ]
          [ .{$x_2 > x_3$}
            % {$x_2$}
            % {$x_3$}
          ]
        ]
        [ .{$x_1 > x_2$}
          [ .{$x_1 > x_3$}
            % {$x_1$}
            % {$x_3$}
          ]
          [ .{$x_2 > x_3$}
            % {$x_2$}
            % {$x_3$}
          ]
        ]
      ]
\end{center}

\subsection*{Code Generation}

After generating a max comparison tree, we can create leaves for our tree that
are different LogSumExp summations. This is rather trivial given a particular
max index.

% We use {\tt log1p} and {\tt expm1} to help prevent.

A Hakaru program that sums 4 probabilities will generate the following C
expression, where {\tt p\_a}, {\tt p\_b}, {\tt p\_c}, and {\tt p\_d} are
variables holding probabilities:

{\small
\begin{verbatim}
p_a > p_b
  ? p_a > p_c
    ? p_a > p_d
      ? p_a + log1p(expm1(p_c - p_a) + (expm1(p_d - p_a) + expm1(p_b - p_a)) + 3)
      : p_d + log1p(expm1(p_b - p_d) + (expm1(p_c - p_d) + expm1(p_a - p_d)) + 3)
    : ( p_c > p_d
      ? p_c + log1p(expm1(p_b - p_c) + (expm1(p_d - p_c) + expm1(p_a - p_c)) + 3)
      : p_d + log1p(expm1(p_b - p_d) + (expm1(p_c - p_d) + expm1(p_a - p_d)) + 3))
  : (p_b > p_c
    ? p_b > p_d
      ? p_b + log1p(expm1(p_c - p_b) + (expm1(p_d - p_b) + expm1(p_a - p_b)) + 3)
      : p_d + log1p(expm1(p_b - p_d) + (expm1(p_c - p_d) + expm1(p_a - p_d)) + 3)
    : (p_c > p_d
      ? p_c + log1p(expm1(p_b - p_c) + (expm1(p_d - p_c) + expm1(p_a - p_c)) + 3)
      : p_d + log1p(expm1(p_b - p_d) + (expm1(p_c - p_d) + expm1(p_a - p_d)) + 3)));
\end{verbatim}
}

We can use {\tt log1p} because where $x_i$ is the maximum
$e^{x_i - \hat x} = 1$. We use {\tt log1p} and {\tt expm1} because they can be
more accurate for small values.

\subsection*{Discussion}

The method presented here and used in HKC for code generation produces an amount
of code that is exponential in the number of arguments. In practice (thus far),
this is not a major issue because {\tt summate} is often used to sum a large
number of probabilities.

\end{document}
